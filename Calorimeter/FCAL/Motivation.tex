 \section{Motivation and Constraints for Forward Calorimetry at Linear Collider Detectors}

 Forward calorimeters at linear collider detectors are indispensable to measure 
 precisely the luminosity, a key quantity needed to convert counting experiments into cross sections.
 The gauge process is low angle Bhabha scattering, which can be calculated with high 
 precision in quantum electrodynamics. Since Bhabha scattering has
 a very steep dependence on the polar angle, an excellent angular resolution is required for its measurement, which poses a challenge to the mechanical precision of the device and the granularity.
 Very forward calorimeters are also foreseen for beam tuning in a fast feedback system to 
 maximize the luminosity during data taking. Last, but not least, forward calorimeters
 increase the angular coverage of the detector, thus improving the missing energy 
 measurement. This is important, e.g. for SUSY models resulting in low momentum particles
 in their decay chain.
 