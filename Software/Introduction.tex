\section{Introduction}
Software for the simulation and reconstruction of detectors at Linear Colliders addresses the following needs of the community:
\begin{itemize}
	\item Studies of detectors with several technology options within the sam e framework.
	\item Reconstruction in highly granular calorimeters with a goal of 3-5\% jet energy resolution.
	\item Flavor tagging of b-, c-, and gluon jets in multi-jet events.
	\item Support of beam tests of detector prototypes as well as large-scale physics and detector simulation studies.
\end{itemize}

The center piece of the Linear Collider software infrastructure is the LCIO event data model and persistence format. The well-controlled evolution of the data format enabled the creation of reconstruction packages that can be shared by all collaborations in the Linear Collider community.

The software packages are well adapted to studying a variety of different detector models, as necessary for thorough detector optimization studies. The detector description facilitates quick changes to the detector layout, such as different sizes or positions for a given sub-detector. At the same time, it allows the addition of new readout schemes, sensitive materials, or geometries.

Starting from simple algorithms borrowed from previous experiments, the reconstruction packages have been incrementally improved on an as-needed basis, in parallel with improvements to the detail of the detector simulation. They now feature sophisticated strategies and state-of-the-art algorithms for pattern recognition and track and vertex reconstruction, and the most advanced particle flow algorithms available.

The following packages have been developed:
\begin{itemize}
	\item LCIO -- Event data model and file format
	\item LCFIPlus -- Vertex reconstruction and flavor tagging
	\item PandoraPFA -- Pattern recognition, calorimeter reconstruction and particle flow algorithms
	\item Marlin -- Event reconstruction framework
	\item DD4HEP -- Detector description for simulation and reconstruction
\end{itemize}
