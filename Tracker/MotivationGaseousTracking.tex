\section{Motivation and Constraints for Gaseous Tracking at Linear Colliders}

Gaseous tracking devices have been extremely successful in providing precision pattern recognition for collider experiments. They provide hundreds of measurements on a single track, with an extremely low material budget in the central region of the detector. This results in excellent pattern recognition and hence tracking efficiency.

The continuous measurements of charged particle tracks also affords particle identification capabilities, which has the possibility to improve the jet energy resolution and flavor tagging capability of the experiment. These are two essential advantages for experiments at a linear collider.

The main challenges for the design of a large Time Projection Chamber (TPC) at a linear collider detector are related to the high magnetic field of the solenoid, in which the detector has to operate. For accurate measurements of the momenta of charged particles, the field map has to be known with high precision. While the event rate of an ILC detector can easily be accommodated by current TPC readout technology, R\&D to mitigate the effects of secondary processes from bunch -- bunch interactions is ongoing.
