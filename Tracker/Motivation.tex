\section{Motivation and Constraints for Silicon Tracking at Linear Colliders}

Accurate matching between the track and the shower components of charged particles is one of the cornerstones of particle flow algorithms. Detector concepts at Linear Colliders therefore aim to measure the tracks of charged particles with the utmost precision to follow them from the point of creation to where they enter the calorimeters.

The concept of Silicon tracking uses several measurements with small errors to reconstruct the helix of charged particles in a magnetic field. The CMS experiment at the LHC is demonstrating that tracking with a precision required by the particle flow paradigm can be successful.

Silicon tracking detectors at Linear Colliders are faced with challenging requirements on the material budget. On the one hand, this goes along with thinner sensors and therefore lower signal-to-noise ratios, on the other hand, this leads to requirements on low power ASICs to allow passive cooling. In addition, silicon tracking allows the precise determination of the time of the hit, which significantly improves pattern recognition in the presence of background from machine-related processes.
