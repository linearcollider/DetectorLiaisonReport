\thispagestyle{empty}
\newgeometry{margin=1.5cm} % modify this if you need even more space
\begin{landscape}
    \centering
    \begin{adjustbox}{max width=1.1\textwidth,totalheight=1\textheight}
\begin{tabularx}{2\textheight}{lXXXX}
    \toprule
    R\&D Technology & Participating Institutes & Description / Concept & Milestones & Future Activities \\
    \midrule
        Asian GEM &
        Saga, KEK, Hiroshima, \newline Kindai, Kogakuin, Iwate, \newline
    Nagasaki IAS, Tsinghua &Design of an endcap readout module with a stack of
    two thicker laser etched polymer-based GEMs and pads  & 2010-2013: Several
    test beam campaigns were performed with three readout modules. & During test beam activities the stability of the HV was not
as good as in lab tests. The origin of the discharges is being
investigated. A modified module with a gating device will be
designed and constructed in the next step, and pad
plane adapted for the SALTRO is also planned. (2016/17) \\
    \midrule
        GEM &
        DESY, Hamburg \newline Bonn \newline Siegen &
        Design of an endcap readout module with a stack
of three standard CERN GEMs and pads&2009-2013: Several test beam campaigns were performed with
three readout modules.

 &Though no problems occurred during the test beam, the
HV-stability is still being investigated. A new module with
gating device, reduced local field distortions and pad
plane adapted for the SALTRO is planned. (2016/17)\\
    \midrule
       Resistive Micromegas &
       CEA Saclay, Carleton &
       Design of an endcap readout module with a Micromegas gas amplification
       stage, a resistive layer for charge dispersion and integrated readout.
       Construction of 11 modules.&
       2010-2015: Several test beam campaigns were performed with up to seven
       readout modules, covering the complete LP-endplate. &
       New materials as resistive layer are being investigated. A module with lower local field distortions is planned.\\
    \midrule
       GridPix Concept \newline GEM + pixel readout &
       Bonn, NIKHEF, CEA Saclay \newline Bonn, Siegen &
       Design of an endcap readout module with a highly pixelized readout with
       GridPixes. These devices consist of a Micromegas mesh built by
       postprocessing technology on a pixel ASIC. Alternatively a GEM-stack is
       used as a gas amplification stage.&
       2009-2015: Several test beam campaigns with up to three modules were
       performed. The three modules featured a total of 160 GridPixes and this
       test beam was performed in March/April 2015. This demonstrated that a
       large area could be covered with GridPixes and about 100 GridPixes per module could be operated.&
       2017: The successor chip Timepix3 will be implemented in the readout chain and
       a test beam with several Timepix-3 chips is planned. This new chips
       fulfills basic requirements for an operation in an ILD environment.\\
    \midrule
        Field cage &
        DESY &
        Design and construction of a TPC field cage &
        2009: A first prototype has been built and is used as a test device at
        DESY &
        A new field cage with improved geometrical precision is
        under construction at DESY. (2017)\\
    \midrule
        Electronics &
        Lund, CERN, CEA Saclay &
        Design of a readout electronics fit for test beam
        operation at T24/1 at DESY and for investigation the
        requirements of the ILD-TPC electronics.&
        2009: Sofar, readout systems based on the AFTER and the
        ALTRO chip have been used with 10,000 channels each.&
        A new system based on the SALTRO-16 chip is being prepared
        (2017). Simulations on the impact of key electronics parameters on the
        TPC-performance are planned.\\
    \midrule
        DAQ &
        Lund, ULB-VUB, Hubei &
        Design of a data acquisition system fit for test beam operation at
        T24/1 at DESY. &
        Sofar, DAQ systems for both readout systems (AFTER and ALTRO) have been set up. &
        A new DAQ system for the SALTRO-16 based readout electronics  is being
        prepared and will be available soon.\\
    \midrule
        Endcap &
        Cornell &
        Study of different endplate designs for an ILD-TPC with
CAD programs and production of smaller endplates fit
for operation at test setup at T24/1 at DESY. &
A detailed model of the endplate was implemented in a
CAD program and in 2009 a first endcap for the test beam
setup has been produced several years ago. A new
version also fulfills the requirements of the material
budget and will be used from 2015 onwards.&
Simulations optimizing the module size are planned.\\
    \midrule
        &
         CEA Saclay, DESY &
        Mechanical studies for ILD-TPC regarding the effect of
pressure, weight, hanging/support schemes on the
mechanical deformation of the endplate and field cage. &
 First studies have been done.
&
 More detailed studies are planned\\
    \midrule
        Calibration &
        BNL, CERN, Indiana, Kolkata &
        Laser calibration system, Alignment/calibration of the TPC, Integration with other tracking systems & & \\
    \midrule
        Study of systematic effects &
        Victoria, Kolkata &
        Field distortions are a major source uncertainties in track
        reconstruction. The sources of these distortions are studied and minimized. & & \\
    \midrule
        Analysis software &
        DESY, Carleton, CEA Saclay, KEK, Saga, Siegen, Tsinghua &
        Development of a software package MarlinTPC, which serves all groups
        for reconstructing and analyzing the test beam data and for
        simulation, reconstruction and analysis of ILD events. &
        MarlinTPC is well developed and the key analysis tool for all
        analyses. &
        Further improvements are continually made.\\
    \midrule
        Ion backflow/Gating &
        Japanese Univers., KEK, Tsinghua, Kolkata, DESY &
        The ion back flow from gas amplification stages is a major source for
        time dependent field distortions and has to be suppressed as much as
        possible. With simulations and experimental setups the minimization of
        ion production and the reduction of the back flow by a gating device
        are under study. Field distortions are a major source uncertainties in track
        reconstruction. The sources of these distortions are studied and
        minimized. & Simulations have shown, that all gas amplification stages
release too many ions into the drift volume and a
gating device is necessary. 2014:  A first MPGD-based device,
a Gating-GEM, has been successfully produced and tested, ruling out the need for a wire gate.
& 2016: module sized GEM-gates will be available.
Then the impact of the gate on the TPC-performance will be
studied by using a UV-laser facility at KEK, and in test
beams.\\
    \midrule
        Cooling &
        KEK, Saga, NIKHEF, Saclay, Kolkata &
        Cooling is important to divert the heat produced by the
readout electronics at the endplate. The temperature
influences the gas gain and drift properties in the
gas and has to be kept as stable as possible to
achieve a reliable measurement. &
2014: A \ce{CO2} cooling plant has been purchased and setup at the
test beam site. First results show a significantly reduced
temperature gradient due to heating at the endplate.
& Cooling of the SALTRO-16 based module will be studied
with a mockup which emulates heat and mechanical
conditions of the module. Then, the SALTRO-16 based
module will be built with the \ce{CO2} cooling pipes.\\
    \bottomrule
\end{tabularx}
\end{adjustbox}
\end{landscape}
\restoregeometry
